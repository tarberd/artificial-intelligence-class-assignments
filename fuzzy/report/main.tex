\documentclass[12pt, a4paper]{article}

\title{Trabalho Prático 4}

\author{Bernardo Ferrari \and Matheus Teixeira}

\date{Novembro 2019}

\begin{document}
\maketitle

\section{Introdução}

Esse trabalho propõem uma solução para a simulação de estacionamento de
caminhões utilizando a biblioteca \textbf{jFuzzy} juntamente a um cliente
java conectado a simulação através de sockets.

\section{Desenvolvimento}

Foi desenvolvido um programa cliente em java que recebe a posição $(x, y)$, $x, y \in [0, 1]$,
acompanhada do angulo $angle \in [0, 360]$ entre o caminhão e o fundo do estacionamento.
É feita a computação da direção do volante do caminhão como saída $wheel \in [-1.0, 1.0]$.

Essa computação é feita através de um sistema fuzzy descrito em `Fuzzy Control Language' (FCL) e
essa descrição é interpretada pela biblioteca java jFuzzy e para executar as computações são
utilizados os seguintes comandos java:

\section{Conclusão}

\end{document}
