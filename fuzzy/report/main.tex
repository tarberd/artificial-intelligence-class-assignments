\documentclass[12pt, a4paper]{article}

\usepackage[
  outputdir=target,
  newfloat=true
]{minted}
\SetupFloatingEnvironment{listing}{name=Código}

\usepackage{indentfirst}

\title{Trabalho Prático 4}
\author{Bernardo Ferrari \and Matheus Teixeira}
\date{Novembro 2019}

\begin{document}
\maketitle

\section{Introdução}

Esse trabalho propõem uma solução para a simulação de estacionamento de
caminhões utilizando a biblioteca \textbf{jFuzzy} juntamente a um cliente
java conectado a simulação através de sockets.

\section{Desenvolvimento}

Foi desenvolvido um programa cliente em java que recebe a posição $(x, y)$, $x, y \in [0, 1]$,
acompanhada do angulo $angle \in [0, 360]$ entre o caminhão e o fundo do estacionamento.
É feita a computação da direção do volante do caminhão como saída $wheel \in [-1.0, 1.0]$.

Essa computação é feita através de um sistema fuzzy descrito em `Fuzzy Control Language' (FCL) e
essa descrição é interpretada pela biblioteca java jFuzzy e para executar as computações é executado
o código java~\ref{listing:jfuzzy}.

\begin{listing}[h!]
  \begin{minted}[breaklines]{java}
String fileName = "src/driver.fcl";
FIS fis = FIS.load(fileName,true);
fis.setVariable("x_pos", x);
fis.setVariable("angle", angle);
fis.evaluate();
double respostaDaSuaLogica = fis.getVariable("wheel").getValue();
  \end{minted}
  \caption{Utilização da biblioteca jFuzzy.}\label{listing:jfuzzy}
\end{listing}

Note que o código~\ref{listing:jfuzzy} lê de um arquivo chamado \textbf{driver.fcl}.
Esse arquivo descreve em FCL o comportamento do sistema fuzzy e inclui as definições
de variáveis de entrada e de saída, definições de \textit{fuzzify} das variáveis de entrada,
definições de \textit{defuzzify} das variáveis de saída e as definições das regras fuzzy.

Para o \textit{fuzzify} da variável $x\_pos$, foram criados conjuntos fuzzy para $center$, $left$ e $right$ utilizando trapézios para definição probabilidades.
Os conjuntos fuzzy para a variável $x\_pos$ são descritos pelo bloco fuzzify descritos no código~\ref{listing:xposfuzzy}.

\begin{listing}[h!]
  \begin{minted}[breaklines]{text}
FUZZIFY x_pos
  TERM center := TRAPE 0.0 0.2 0.8 1.0;
  TERM left := TRAPE -0.1 0 0.2 0.5;
  TERM right := TRAPE 0.5 0.8 1 1.1;
END_FUZZIFY
\end{minted}
  \caption{Conjuntos fuzzy para variável de entrada $x\_pos$.}\label{listing:xposfuzzy}
\end{listing}

Para o \textit{fuzzify} da variável $angle$, foram criados conjuntos fuzzy para $up$, $right$, $left$ e $down$ com sobreposição entre os trapézios.
Além de $up\_left$, $left\_down$, $down\_right$ e $right\_up$, para melhor controle das regiões de intersecção dos conjuntos $up$, $right$, $left$ e $down$.
Os conjuntos fuzzy para a variável $angle$ são descritos pelo bloco fuzzify descritos no código~\ref{listing:anglefuzzy}.

\begin{listing}[h!]
  \begin{minted}[breaklines]{text}
FUZZIFY angle
  TERM up         := TRAPE 0 80 100 180;
  TERM up_left    := TRAPE 90 120 150 180;
  TERM left       := TRAPE 90 170 190 270;
  TERM left_down  := TRAPE 180 210 240 270;
  TERM down       := TRAPE 180 260 280 360;
  TERM down_right := TRAPE 270 300 330 360;
  TERM right      := (0, 1) (2.5, 1) (90, 0) (270, 0) (357.5, 1) (360, 1);
  TERM right_up   := TRAPE 0 30 60 90;
END_FUZZIFY
  \end{minted}
  \caption{Conjuntos fuzzy para variável de entrada $angle$.}\label{listing:anglefuzzy}
\end{listing}

Para o \textit{defuzzify} da variável de saída $wheel$, foram criados conjuntos para $straight$, $turn\_left$ e $turn\_right$ com sobreposição entre os trapézios.
Foi usado o método de centro de gravidade COG para defuzzificação da variável de valor padrão $default = 0$.
Os conjuntos fuzzy para a variável $wheel$ são descritos pelo bloco defuzzify descritos no código~\ref{listing:wheelfuzzy}.

\begin{listing}[h!]
  \begin{minted}[breaklines]{text}
DEFUZZIFY wheel
  TERM straight := TRAPE -0.5 -0.1 0.1 0.5;
  TERM turn_left := TRAPE -1.1 -1 -0.5 0;
  TERM turn_right := TRAPE 0 0.5 1 1.1;
  METHOD : COG;
  DEFAULT := 0;
END_DEFUZZIFY
  \end{minted}
  \caption{Conjuntos fuzzy para variável de saída $wheel$.}\label{listing:wheelfuzzy}
\end{listing}

\section{Conclusão}

\end{document}
